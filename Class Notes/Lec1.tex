\documentclass[11pt]{article}
\usepackage[utf8]{inputenc}	% Para caracteres en español
\usepackage{amsmath,amsthm,amsfonts,amssymb,amscd}
\usepackage{multirow,booktabs}
\usepackage[table]{xcolor}
\usepackage{fullpage}
\usepackage{lastpage}
\usepackage{enumitem}
\usepackage{fancyhdr}
\usepackage{mathrsfs}
\usepackage{wrapfig}
\usepackage{setspace}
\usepackage{calc}
\usepackage{multicol}
\usepackage{cancel}
\usepackage[retainorgcmds]{IEEEtrantools}
\usepackage[margin=3cm]{geometry}
\usepackage{amsmath}
\newlength{\tabcont}
\setlength{\parindent}{0.0in}
\setlength{\parskip}{0.05in}
\usepackage{empheq}
\usepackage{framed}
\usepackage[most]{tcolorbox}
\usepackage{xcolor}
\usepackage[hidelinks]{hyperref}
\colorlet{shadecolor}{orange!15}
\parindent 0in
\parskip 12pt
\geometry{margin=1in, headsep=0.25in}
\theoremstyle{definition}
\newtheorem{defn}{Definition}
\newtheorem{reg}{Rule}
\newtheorem{exer}{Exercise}
\newtheorem{note}{Note}
\usepackage{mathrsfs}

\begin{document}

%Change this for headings
\setcounter{section}{0}
\title{Lecture 1 Class Notes}

\thispagestyle{empty}

\begin{center}
{\LARGE \bf Class Notes| Lecture 1}\\
{\large MSO: Introduction to Probability Theory}\\
Fall 2024
\end{center}
%Heading Ends


\tableofcontents


%Content Starts
\section{Definitions}
\begin{note}
\textbf{Definition 1} (Random Experiment): An experiment in which:
\begin{itemize}
\item all possible outcomes of the experiment are known in advance
\item outcome of a particular trial/performance of the experiment cannot be specified in advanced
\item the experiment can be repreated under identical conditions
\end{itemize}
Denoted via $\varepsilon$
\end{note}

\begin{note}
{\textbf{Definition 2}} (Sample Space): THe collection of all possible outcomes of the random experiment $\varepsilon$ is called its sample space. Example:
$$
\omega = {(x,y,z): x,y,z \in {H, T}}
$$

\end{note}


\begin{note}
{\textbf{Definition 3}} (Events): If the outcome of a random experiment $\varepsilon$ is an element of a subset $\mathscr{E}$ of $\omega$, then we say that the event $\mathscr{E}$ has occured. An event is an set object. 

Collection of all events is denoted as $\mathscr{F}$. This means $\mathscr{F}$ is a set of sets. Empty Set $\emptyset$ and the sample space $\Omega$ will always be an element in $F$.
\end{note}

\begin{note}
{\textbf{Definition 4}} (Probability, Classical (A priori) Definition): Suppose that a random experiment results in $n$ (a finite number) outcomes. Given an event $mathscr{A} \in \mathscr{F}$, if it apears in m ($0 \leq m \leq n$) outcomes, then the probability of $\mathscr{A}$ is $\frac{m}{n}$. \
\end{note}

The classical definition only works when there are finitely many outcomes. Due to the limitations of this definition, we look for other ways to understand the notion of probability.

\begin{note}
{\textbf{Definition 5}} (Probability, Relative Frequency(A posteriori) Definition): If a random experiment $\mathscr{E}$ is repeared a large number, say $n$, of times and an event $\mathscr{A}$ occurs $m$ many times, then the relative frequency $\frac{m}{n}$ may be taken as an approximate value of a probability of $\mathscr{A}$.
\end{note}


\end{document}
