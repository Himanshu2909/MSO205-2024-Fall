\documentclass[11pt]{article}
\usepackage[utf8]{inputenc}	% Para caracteres en español
\usepackage{amsmath,amsthm,amsfonts,amssymb,amscd}
\usepackage{multirow,booktabs}
\usepackage[table]{xcolor}
\usepackage{fullpage}
\usepackage{lastpage}
\usepackage{enumitem}
\usepackage{fancyhdr}
\usepackage{mathrsfs}
\usepackage{wrapfig}
\usepackage{setspace}
\usepackage{calc}
\usepackage{multicol}
\usepackage{cancel}
\usepackage[retainorgcmds]{IEEEtrantools}
\usepackage[margin=3cm]{geometry}
\newlength{\tabcont}
\setlength{\parindent}{0.0in}
\setlength{\parskip}{0.05in}
\usepackage{empheq}
\usepackage{framed}
\usepackage[most]{tcolorbox}
\usepackage{xcolor}
\usepackage[hidelinks]{hyperref}
\colorlet{shadecolor}{orange!15}
\parindent 0in
\parskip 12pt
\geometry{margin=1in, headsep=0.25in}

\theoremstyle{definition}
\newtheorem{defn}{Definition}
\newtheorem{reg}{Rule}
\newtheorem{exer}{Exercise}
\newtheorem{note}{Note}
\usepackage{mathrsfs}
\newtheorem{theorem}{Theorem}[section]
\newtheorem{prop}{Proposition}[section]
\newtheorem{corollary}{Corollary}[theorem]
\newtheorem{lemma}[theorem]{Lemma}
\renewcommand\qedsymbol{QED}


\begin{document}

%Change this for headings
\setcounter{section}{0}
\title{Week 1 Class Notes}

\thispagestyle{empty}

\begin{center}
{\LARGE \bf Class Notes| Week 1}\\
{\large MSO: Introduction to Probability Theory}\\
Fall 2024
\end{center}
%Heading Ends


\tableofcontents


%Content Starts
\section{Basics of Probability}
\subsection{Axioms and Definitions}

\begin{defn}[Random Experiment] An experiment in which:
\begin{itemize}
\item all possible outcomes of the experiment are known in advance
\item outcome of a particular trial/performance of the experiment cannot be specified in advanced
\item the experiment can be repreated under identical conditions
\end{itemize}
Denoted via $\varepsilon$
\end{defn}

\begin{defn}
[Sample Space] The collection of all possible outcomes of the random experiment $\varepsilon$ is called its sample space. Example, for the random experiment of tossing three coins at the same time, the sample space shall be:
$$
\Omega = \{(x,y,z): x,y,z \in {H, T}\}
$$
\end{defn}

\begin{defn}
[Events] If the outcome of a random experiment $\varepsilon$ is an element of a subset $\mathcal{E}$ of $\Omega$, then we say that the event $\mathcal{E}$ has occured. An event is an set object.

Collection of all events is denoted as $\mathcal{F}$ and is called the \textbf{Event Space}  This means $\mathcal{F}$ is a set of sets. Empty Set $\emptyset$ and the sample space $\Omega$ will always be an element in $\mathcal{F}$.$\mathcal{F}$ is the power set of $\Omega$.
\end{defn}


\begin{defn} [Event Space]
We say that $\mathcal{F}$ is an event space if:
\begin{itemize}
    \item[1] $\Omega \in \mathcal{F}$
    \item[2] If $A \in \mathcal{F}$, then $A' \in \mathcal{F}$
    \item[3] If $A_1, A_2, ..., A_n \in \mathcal{F}$, then $A_1 \cup A_2 \cup ... \cup A_n \in \mathcal{F}$ 
\end{itemize}
\end{defn}

\begin{defn}
[Probability, Classical (A priori) Definition] Suppose that a random experiment results in $n$ (a finite number) outcomes. Given an event $\mathit{A} \in \mathcal{F}$, if it apears in m ($0 \leq m \leq n$) outcomes, then the probability of $\mathit{A}$ is $\frac{m}{n}$. \
\end{defn}

The classical definition only works when there are finitely many outcomes. Due to the limitations of this definition, we look for other ways to understand the notion of probability.\\

\begin{defn}[Probability, Relative Frequency(A posteriori) Definition]
If a random experiment $\mathcal{E}$ is repeared a large number, say $n$, of times and an event $\mathit{A}$ occurs $m$ many times, then the relative frequency $\frac{m}{n}$ may be taken as an approximate value of a probability of $\mathit{A}$.

The a posteriori definition of probability works only after performing the random experiment.
\end{defn}


\begin{defn}[Set Function]
A set function is a function whose domain is a collection/class of sets
\end{defn}

\begin{defn}[Probability functiion/measure]
Suppose that $\Omega$ and $\mathcal{F}$ are the sample space and the event event space of an event $\mathcal{E}$ repspectively. A real valued set function $\mathbb{P}$, defined on event space $\mathcal{F}$, is said to be the Probability function/measure if it satisfies the following properties:
\begin{itemize}
\item $\mathbf{P}(\emptyset) = 0$
\item (non-negativity) $\mathbf{P}(E) \geq 0$ for any event $E$ in $\mathcal{F}$
\item (Countable additivity) If $\{E_n\}_n$ is a sequence of events in $\mathcal{F}$ such that $E_i \bigcap E_j = \emptyset , \forall i \neq j$, then $\mathbb{P} (\bigcup_{n=1}^{\infty} E_n) = \sum_{n = 1}^{\infty} \mathbb{P}(E_n)$.
\end{itemize}
\end{defn}

\begin{defn}[Probability Space]
If $\mathbb{P}$ is a probability function defined on the event space $\mathcal{F}$ of a random experiment $\mathcal{E}$, then the triple ($\Omega$ , $\mathcal{F}$, $\mathbb{P}$) is said to be a probability space. Here, $\Omega$ denotes the sample space of $\mathcal{E}$.
\end{defn}

\begin{defn}[Mutually Exclusive/Pairwise Disjoint Events]
Let $\mathcal{I}$ be an Indexing set. A collection of events $\{E_i: i \in \mathcal{I}\}$ is said to be mutually exclusive or pairwise disjoint if $E_i \bigcap E_j = \emptyset, \forall i \neq j$.
\end{defn}

\subsection{Primary Propositions}

Let $(\Omega, \mathcal{F}, \mathbb{P})$ be a probability space associated with a random experiment $\mathcal{E}$.

\begin{prop}
$\mathbb{P}(\emptyset) = 0$.
\end{prop}
\begin{proof}
Consider an infinite sequence of events $\{E_i\}_\infty$ wherein $E_1 = \Omega$, and $E_i \forall i \geq 2 = \emptyset$. We know that $E_i \bigcap E_j = \emptyset , \forall i \neq j$. Hence, by definition, we have:
\begin{align*}
&\mathbb{P} (\bigcup_{n=1}^{\infty} E_n) = \sum_{n = 1}^{\infty} \mathbb{P}(E_n) \\
&\implies 1 = 1+\sum_{n = 2}^{\infty} \mathbb{P}(E_n) \\
&\implies 0 = \lim_{k \rightarrow \infty} \sum_{n = 2}^{k} \mathbb{P}(E_n) \\
&\implies 0 = \lim_{k \rightarrow \infty} (k-1) \mathbb{P}(\emptyset) \\
&\implies \mathbb{P}(\emptyset) = 0
\end{align*}
\end{proof}

\begin{prop}[Finite Additivity]
Let $E_1, E_2 ... , E_n \in \mathcal{F}$ for some integer $n \geq 2$ be \textbf{mutually exclusive} events. Then $\mathbb{P} (\bigcup_{i=1}^n E_i) = \sum_{i = 1}^n \mathbb{P}(E_i)$.
\end{prop}
\begin{proof}
Hint: Take all elements after the $n^{th}$ elements to be $\emptyset$
\end{proof}


\begin{prop}
$\mathbb{P}(E) + \mathbb{P}(E^c) = 1$ for all events $E \in \mathcal{F}$.
\end{prop}
\begin{proof}
Hint: The two events are mutually exclusive and their union is $\Omega$
\end{proof}

\begin{prop}
$0\leq \mathbb{P}(E) \leq 1 \forall$ events $E \in \mathcal{F}$.
\end{prop}
\begin{proof}
Hint: Lower limit via definition, upper limit via Proposition 1.3
\end{proof}

\begin{prop}[Monotonicity]
Suppose $A, B \in \mathcal{F}$, and $A \subseteq B$, then $\mathcal{P}(A) \leq \mathcal{P}(B)$
\end{prop}
\begin{proof}
Hint: $\mathbb{P}(B) = \mathbb{P}(A) + \mathbb{P}(A^c \bigcap B)$, also mention finite additivity and the fact that $A$ and  $A^c \bigcap B$ are mutually exclusive.
\end{proof}

\begin{prop}[Inclusion-Exclusion principle for two events]
For $A, B \in \mathcal{F}$, we have:
$$
\mathbb{P}(A \bigcup B) = \mathbb{P}(A) + \mathbb{P}(B) - \mathbb{P}(A \bigcap B)
$$
\end{prop}
\begin{proof}
Hint: Divide the sets into parts, show mutual exclusivity, hence use Finite Additivity to prove.
\end{proof}

\begin{prop}[Boole's inequality for two events]
For $A, B \in \mathcal{F}$, we have: $\mathbb{P}(A) + \mathbb{P}(B) \geq \mathbb{P}(A \bigcup B)$
\end{prop}
\begin{proof}
Hint: Follows from Inclusion-Exclusion.
\end{proof}

\begin{prop}[Bonferroni's inequality for two events]
For $A, B \in \mathcal{F}$, we have: $\mathbb{P}(A \bigcup B) \geq max\{0, \mathbb{P}(A)+\mathbb{P}(B)-1\}$
\end{prop}
\begin{proof}
Hint: Follows from Inclusion-Exclusion and boundary condition of $\mathbb{P}(E)$
\end{proof}


\begin{defn}[Rigorous definition of Probability Function] Let $\Omega$ be any finite or countably infinite set. Consider $\mathcal{F} = 2^{\omega}$ the power set. let $p: \Omega \rightarrow [0, 1]$ be a function such that:
$$
\sum_{\omega \in \Omega} p_{\omega} = 1
$$
Now consider a real valued set function $\mathbb{P}$ on $\mathcal{F}$ defined by:
$$
\mathbb{P}(\mathbb{A}) = \sum_{\omega \in \mathbb{A}} p_{\omega}
$$
\end{defn}

\begin{prop}
The function $\mathbb{P}$  defined above is the probability function on $\mathcal{F}$.
\end{prop}
\begin{proof}
Hint: Verify all the 3 definitions for the function defined. 
\end{proof}

\begin{defn}[Discrete Probability spaces]
Let $\Omega$ be a finite or countable set. We refer to a probability space of the form $(\Omega, 2^{\Omega} , \mathbb{P})$ as a discrete probability space.
\end{defn}

\begin{defn}[Elementary Events]
We may refer to the singleton events in a discrete probability space as elementary events.
\end{defn}

\begin{prop}[Generalized Inclusion-Exclusion Principlle]
Let $(\Omega, \mathcal{F}, \mathbb{P})$ define a probability space and let $A_1,...,A_n$ be the events. Then:
$$
\mathbb{P}(\bigcap_{i=1}^n A_i) = S_{1,n}-S_{2,n}+S_{3,n}-...+{-1}^{n-1}S_{n,n},
$$
where
$$
S_{n,k} := \sum _{1\leq i_1 <i_2<...<i_k\leq n} \mathbb{P}(A_{i_1} \cup A_{i_2} \cup ... \cup A_{i_n})
$$
\end{prop}
\begin{proof}
Hint: Already proven for $n=2$. Use induction  now.
\end{proof}

\end{document}
